\ResetPicDir{}
\SetPicSubDir{fyp_results}

\section{Circuit Noise Characterisation}
% Simulation parameter
% Monte Carlo
% equivalence in distributions for resistance & conductance

Due to variations and non-ideal behaviour in the analog domain, there will be a deviation from expected results which will be defined as noise.
To characterise the noise behaviour of the CBA and the neuron circuit, a Monte Carlo analysis with Gaussian distribution at the input.
The results at the output is observed, and this is viewed as a distribution of noise about an expected output. 

\subsection{Noise from crossbar array}
% Noise from crossbar array
% Mathematical analysis of scaling up the CBA

To understand the CBA noise relationship, CBAs of various sizes was generated and the error in the result of the CBA output was computed. The Monte Carlo simulation parameters included the memristor cell-level standard deviation and the CBA size. It was found that the CBA output result distribution maintains a Gaussian shape even as the CBA size is scaled up. The simulation was repeated to form a distribution of results, where the mean and standard deviation were measured. This allows for visualization of the relationship between CBA size, and the mean (\autoref{results:fig:chart_cba_mean_relationship}) or standard deviation (\autoref{results:fig:chart_cba_sd_relationship}).

\begin{figure}[!ht]
    \centering
    \includesvg[width=\linewidth]{\Pic{svg}{chart_cba_mean_relationship}}
    \parbox{0.7\linewidth}{\footnotesize \emph{\underline{Units:} \\ CBA Size (x-axis): no. of wordlines [dimensionless] \\ Mean (y-axis): current magnitude [amps]}}
    \caption{Behaviour of Mean as {CBA} Size increases}
    \label{results:fig:chart_cba_mean_relationship}
\end{figure}%

\begin{figure}[!ht]
    \centering
    \includesvg[width=\linewidth]{\Pic{svg}{chart_cba_sd_relationship}}
    \parbox{0.7\linewidth}{\footnotesize \emph{\underline{Units:} \\ CBA Size (x-axis): no. of wordlines [dimensionless] \\ Standard deviation (y-axis): current magnitude [amps]}}
    \caption{Behaviour of Standard Deviation as {CBA} Size increases}
    \label{results:fig:chart_cba_sd_relationship}
\end{figure}

It is observed that the mean ($\mu$) increases linearly as the CBA size (k) increases, and the CBA standard deviation ($\sigma_{CBA}$) increases with the square-root of the CBA size ($\sqrt{k}$).

\noindent
From a mathematical analysis, the formula to calculate the CBA standard deviation ($\sigma_{CBA}$) was found as follows:

\vspace{0.5em}
\setlength{\fboxsep}{0.2em}
\noindent
\centerline{\framebox[0.90\linewidth][l]{
    \parbox{0.85\linewidth}{
        \centering
        $$ \Rightarrow \sigma_{CBA} = \frac{\sqrt{k} \cdot C}{\mu_{R}} $$

        \footnotesize
        where $k$ is the number of rows of the crossbar, \\
        $\mu_{R} = R$ is the nominal resistance of the memristor cell, \\
        and 
        $C = \sfrac{\sigma_R}{\mu_R}$ is the relative standard deviation.
        \vspace{0.5em}
    }
}}
\vspace{0.25em}

\subsection{Noise from neuron circuitry}
% Noise from neuron circuitry

In the neuron circuitry, noise is introduced due to variability in CMOS and passive components in the Current-to-Voltage Conversion topology. The desired behavior is a conversion ratio of \verb|1mA:1V|. 
Based on Monte Carlo simulations, the circuit exhibits a mean ($\mu$) scaling factor of \verb|1mA:0.953V| with a standard deviation ($\sigma$) of $4.0478 \times 10^{-2}$ V.
The PDF plot of the output voltage obtained in \autoref{results:fig:i2v_converter_dist} is from a Monte Carlo setting of 5\% standard deviation for all values of passive components and conductance parameter of the MOSFETs.

\begin{figure}[!ht]
    \centering
    \includegraphics[width=0.75\linewidth]{\Pic{png}{i2v_converter_dist}}
    \caption{Probability density function (PDF) plot for the Monte Carlo simulation of the current-to-voltage converter output}
    \label{results:fig:i2v_converter_dist}
\end{figure}

%%%%%%%%%%%%%%%%%%%%%%%%%%%%%%%%%%%%%%%%%%%%%%%%%%%%%%%%%%%%%%%%%%%%%%%%%%%%
\section{Projected Memristor Bit Limitations}
%  Modelling the theoretical limit:
% projected maximum bits for hardware architecture:

There is also a theoretical limit to the number of bits that can be mapped to a memristor cell as we scale the {CBA}. From (\autoref{results:fig:chart_cba_sd_relationship}), the distribution of error becomes larger as the CBA size increases. A larger error means that the quantisation step size increases and number of quantisation levels that can be mapped to a memristor decreases.

\noindent
A projection of the theoretical maximum number of bits is presented in \autoref{results:table:max_bits}. The memristor states are set to $R_{HRS} = 11 k\Omega$ and $R_{LRS} = 500 \Omega$ as per the device parameters by \citet{PengGu2015} \cite{PengGu2015} for direct comparison. A cell-level standard-deviation $\sigma = (15\% \times \mu)$ was chosen with an acceptance criteria of $2\sigma$.

\begin{table}[ht!]
    \addtolength{\tabcolsep}{-0.5em} % reduce the spacing between columns in a table
    \centering
    \resizebox{\linewidth}{!}{%
    \begin{tabular}{@{}ccccc@{}}
    \textbf{CBA Size} & \textbf{SD of CBA Result {[}A{]}} & \textbf{Quantization levels} & \textbf{No. of bits} & \textbf{Max no. of bits} \\ 
    \Xhline{3\arrayrulewidth}
    \noalign{\vskip 0.3em}
    1 & 1.17E-06 & 427.7777778 & 8.74 & 8 \\
    2 & 1.65E-06 & 302.4845675 & 8.24 & 8 \\
    4 & 2.34E-06 & 213.8888889 & 7.74 & 7 \\
    8 & 3.31E-06 & 151.2422838 & 7.24 & 7 \\
    %12 & 4.05E-06 & 123.4888076 & 6.95 & 6 \\
    16 & 4.68E-06 & 106.9444444 & 6.74 & 6 \\
    %24 & 5.73E-06 & 87.31977324 & 6.45 & 6 \\
    32 & 6.61E-06 & 75.62114188 & 6.24 & 6 \\
    48 & 8.10E-06 & 61.74440379 & 5.95 & 5 \\
    \rowcolor{yellow} 50 & 8.26E-06 & 60.4969135 & 5.92 & 5 \\
    64 & 9.35E-06 & 53.47222222 & 5.74 & 5 \\
    80 & 1.05E-05 & 47.82700952 & 5.58 & 5 \\
    96 & 1.15E-05 & 43.65988662 & 5.45 & 5 \\
    128 & 1.32E-05 & 37.81057094 & 5.24 & 5 \\
    256 & 1.87E-05 & 26.73611111 & 4.74 & 4 \\
    384 & 2.29E-05 & 21.82994331 & 4.45 & 4 \\
    512 & 2.64E-05 & 18.90528547 & 4.24 & 4 \\
    %640 & 2.96E-05 & 16.90940138 & 4.08 & 4 \\
    768 & 3.24E-05 & 15.43610095 & 3.95 & 3 \\
    1024 & 3.74E-05 & 13.36805556 & 3.74 & 3 \\
    4096 & 7.48E-05 & 6.684027778 & 2.74 & 2 \\
    % 100 & 1.17E-05 & 42.77777778 & 5.42 & 5 \\
    % 768 & 3.24E-05 & 15.43610095 & 3.95 & 3
    \hline
    \end{tabular}%
    }
    \caption{Theoretical maximum number of bits for a given CBA size. Cell-level $\sigma = (15\% \times \mu)$ and acceptance criteria of $2\sigma$ was chosen.}
    \label{results:table:max_bits}
\end{table}

\vspace{-0.5em}

\noindent
These are the main parameters affecting the theoretical limit as we scale the CBA:

\begin{itemize}
    \item \textbf{Cell-level Standard Deviation:} Depends on the manufacturing process \& technology which causes the variability in the memristor high- and low-resistance states.
    \item \textbf{Allowable $\sigma$ range:} Determines how strict the acceptance criteria for a manufactured {CBA} is. For a $2\sigma$ acceptance criteria, about 95.44\% of the CBAs manufactured are within the specifications.
\end{itemize}

\vspace{0.5em}
\noindent
There are some simplifications in the model as it neglects several factors including:

\begin{itemize}
    \item \textbf{RRAM interconnect resistance:} Assumed to be zero because interconnect resistance is much smaller than memristor cell resistance.
    
    \item \textbf{{CBA} loading resistance:} Also assumed to be minimal since the current-to-voltage converter is designed to minimise loading effect.

    \item \textbf{ADC/DAC conversion speed:} The maximum conversion rate and conversion noise are also not considered as it is assumed that there are no constraints for settling time of the output result.
\end{itemize}

\vspace{0.5em}

\noindent
\emph{Note that due to the simplifications above, the projected numbers presented in \autoref{results:table:max_bits} are the optimistic upper limits.}

\vspace{0.5em}
For a state-of-the-art comparison, researchers from Tsinghua University produced a 4-bit {CBA} with a size of $50\times 50$ for a MNIST classifier accelerator with device characteristic of $R_{HRS} = 11 k\Omega$ and $R_{LRS} = 500 \Omega$ \cite{PengGu2015}. As highlighted in \autoref{results:table:max_bits}, the theoretical maximum is 5-bits for a size-50 {CBA}. Therefore, it shows that \citet{PengGu2015} \cite{PengGu2015} are working very closely to the technical limitations since they have achieved 4-bits in their work.

%%%%%%%%%%%%%%%%%%%%%%%%%%%%%%%%%%%%%%%%%%%%%%%%%%%%%%%%%%%%%%%%%%%%%%%%%%%%
\section{Hardware sigmoid limitations}

\begin{figure}[!ht]
    \centering
    \begin{minipage}[b]{0.48\linewidth}
        \includesvg[width=\linewidth]{\Pic{svg}{sigmoid_fx}}
        \caption{Comparison of the hardware sigmoid to the software sigmoid of $\beta=8$}
        \label{results:fig:sigmoid_fx}
    \end{minipage}%
    \hspace{0.02\linewidth}
    \begin{minipage}[b]{0.48\linewidth}
        \includesvg[width=\linewidth]{\Pic{svg}{sigmoid_grad}}
        \caption{Comparison of the gradients of the hardware and software sigmoid of $\beta=8$}
        \label{results:fig:sigmoid_grad}
    \end{minipage}%
\end{figure}

The voltage-transfer graph of the hardware implementation of the sigmoid activation function is shown in \autoref{results:fig:sigmoid_fx}. The hardware sigmoid closely resembles the software sigmoid function $\displaystyle f(x) = \frac{1}{1+e^{-\beta x}}$ with $\beta = 8$. 

\vspace{0.5em}

An important issue is that the gradient of the hardware sigmoid at $x = 0$ is almost vertical compared to the software sigmoid which has a more gentle slope. Seen in \autoref{results:fig:sigmoid_grad}, it gives rise to a gradient that very similarly approaches a dirac-delta impulse function. This makes the backpropagation a tricky operation in the analog domain, because the sharp increase in the magnitude may cause clipping if the result exceeds the supply voltage rails.

\subsection{Performing backpropagation with psuedo-gradients}

To circumvent the issue of the large gradient magnitude, psuedo-gradients can be used for backpropagation instead. Psuedo-gradient are commonly used as straight-through estimators (STEs) for non-differentiable activation functions. \autoref{results:fig:sigmoid_grad_overall} shows that the gradient of the sigmoid converges to a rectangle function at low bit-precisions.

\begin{figure}[!ht]
    \begin{minipage}[b]{0.49\linewidth}
        \begin{subfigure}{\linewidth}
            \centering
            \includesvg[width=\linewidth]{\Pic{svg}{sigmoid_grad}}
            \caption{Without quantisation}
            \label{results:fig:sigmoid_grad_a}
        \end{subfigure}
    \end{minipage}%
    \begin{minipage}[b]{0.49\linewidth}
        \begin{subfigure}{\linewidth}
            \centering
            \includesvg[width=\linewidth]{\Pic{svg}{sigmoid_grad_qnt100}}
            \caption{12-bit quantisation}
            \label{results:fig:sigmoid_grad_b}
        \end{subfigure}
    \end{minipage}%

    \vspace{0.5em}

    \begin{minipage}[b]{0.49\linewidth}
        \begin{subfigure}{\linewidth}
            \centering
            \includesvg[width=\linewidth]{\Pic{svg}{sigmoid_grad_qnt200}}
            \caption{8-bit quantisation}
            \label{results:fig:sigmoid_grad_c}
        \end{subfigure}
    \end{minipage}%
    \begin{minipage}[b]{0.49\linewidth}
        \begin{subfigure}{\linewidth}
            \centering
            \includesvg[width=\linewidth]{\Pic{svg}{sigmoid_grad_qnt250}}
            \caption{6-bit quantisation}
            \label{results:fig:sigmoid_grad_d}
        \end{subfigure}
    \end{minipage}%

    \caption{Effects of quantisation on the gradient of the sigmoid}
    \label{results:fig:sigmoid_grad_overall}

\end{figure}

% \begin{figure}[!ht]
%     \centering
%     \includegraphics[width=0.70\linewidth]{\Pic{png}{sigmoid_ste}}
%     \caption{Usage of a {STE} for sigmoid function}
%     \label{results:fig:sigmoid_ste}
% \end{figure}

%%%%%%%%%%%%%%%%%%%%%%%%%%%%%%%%%%%%%%%%%%%%%%%%%%%%%%%%%%%%%%%%%%%%%%%%%%%%
\section{Noise sensitivity analysis}

\begin{figure}[!ht]
    \centering
    \includegraphics[width=1.0\linewidth]{\Pic{png}{noisy_bnn_model}}
    \caption{Noisy BNN architecture of $100\times 50\times 10$. The noisy layers are cascaded
    after the weights}
    \label{results:fig:noisy_bnn_model}
\end{figure}

A MNIST classifier model is initially trained for a BNN architecture with dimensions of $100\times 50\times 10$. Noisy layers are then introduced to observe the effect on test accuracy. \autoref{results:fig:trisurf_1} and \autoref{results:fig:trisurf_2} show the relationship between accuracy against the two parameters that affect the crossbar noise distribution.

\vspace{0.25em}

\noindent
The observations are as follows:

\begin{enumerate}
    \item Test accuracy decreases as cell-level standard deviation ($\displaystyle{ \sfrac{\sigma}{\mu} }$) increases.
    \item Test accuracy decreases as the ratio of the high and low resistance state ($\displaystyle{ \sfrac{R_{HRS}}{R_{LRS}} }$) decreases.
\end{enumerate}

\vspace{0.5em}

\noindent
This analysis helps designers foresee the allowable accuracy loss of the MNIST model based on the resistive limitations of a given memristor device technology and determine the manufacturing acceptance criteria to keep performance at an acceptable level. 
For instance, consider an allowable accuracy loss of 2\%, with a memristor device with $\displaystyle{ \sfrac{R_{HRS}}{R_{LRS}} = \sfrac{11k\Omega}{500\Omega} = 22 }$. From \autoref{results:fig:trisurf_2}, the allowable cell-level standard deviation must be less than 20\% to maintain acceptable levels of performance.

\begin{figure}[!ht]
    \centering
    \includegraphics[width=1.0\linewidth]{\Pic{png}{trisurf_1}}
    \caption{3D Plot showing the relationship between test error, resistance state ratio and cell variations}
    \label{results:fig:trisurf_1}
\end{figure}

\begin{figure}[!ht]
    \centering
    \includegraphics[width=1.0\linewidth]{\Pic{png}{trisurf_2}}
    \parbox{1.0\linewidth}{\centering \footnotesize \emph{Note: Black lines are 1\% intervals. Red lines are 5\% intervals.}}
    \caption{Plot loss of accuracy on the crossbar array.}
    \label{results:fig:trisurf_2}
\end{figure}
