\section{Conclusions}

This paper presents a framework to evaluate of the losses in the MNIST model test accuracy when model weights and derivation of neuron circuitry with configurable activation functions. This allows the author do a projection for the proposed hardware with consideration of the theoretical maximum limits of bit precision. The results suggest that other authors' such as \citet{PengGu2015} \cite{PengGu2015} are also facing similar constraints as they work very closely to theoretical limits of memristor CBAs. An important learning point is the difficulty in minimising the loading effect on the CBA in designing the neuron circuitry when working with low currents. Finally, the noise sensitivity analysis was used to evaluate the feasibility of scaling up the CBA for larger networks while maintaining acceptable performance levels. The analysis 
allows for designers forecast the allowable accuracy loss of the MNIST model given the resistive constraints of their memristor device, and decide on the manufacturing acceptance criteria.
% Future research work for the accelerator hardware include supporting other neural network operations such as batch normalisation and convolution layers, consideration for CBA area complexity and FD-SOI technology for low-power design of the neuron circuitry.

%%%%%%%%%%%%%%%%%%%%%%%%%%%%%%%%%%%%%%%%%%%%%%%%%%%%%%%%%%%%%%%%%%%%%%%%%%%%
\section*{Acknowledgment}

The author would like to thank his supervisor, Assistant Professor Fong Xuanyao Kelvin, for his patience and guidance throughout this research project.

%%%%%%%%%%%%%%%%%%%%%%%%%%%%%%%%%%%%%%%%%%%%%%%%%%%%%%%%%%%%%%%%%%%%%%%%%%%%
% Force all items in reference list to show up
\phantom{
    \tiny\cite{CAMPBELL201710,Ielmini_2016,Joksas_Mehonic_2020,Molinar_2016,Strubell_Ganesh_McCallum_2019,Strukov_Snider_Stewart_Williams_2008,Sun_Ielmini_2022,Zou_Xu_Chen_Yan_Han_2021,Crafton_West_Basnet_Vogel_Raychowdhury_2019,iEECON2022,PengGu2015}
}
