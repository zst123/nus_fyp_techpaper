\ResetPicDir{}

\section{Introduction}
% no \PARstart

If you have an introduction for your paper, put it here.
This sample file is intended to serve as a ``starter file."
You need to cut out our text and then insert your text into this file.

\subsection{Subsection Heading Here}
Note that you need to use $\backslash$subsection.
Subsection text goes here, if applicable.
You may or may not have any subsections. That is OK.

\subsubsection{Subsubsection Heading}
Insert subsubsection text here.
Same thing, you may or may not have any subsubsections. That is fine.

\subsubsection{About This Template}
This sample paper is for latex users. Authors may use the sample paper 
here to produce their own paper.
WORD users can also download the template file for WORD posted on the
CEC 2007 website.

\subsection{Page layout}
\begin{itemize}
\item IEEE now only accepts 100$\%$ 
	Xplore compliant papers prepared in PDF format. Please make sure that 
	you follow these guidelines in preparing 
	your PDF files. Violations of any of these specifications 
	may result in rejection of your papers.  
\item Paper size: US letter format ($8.5\times 11$ in) 
or $216\times 278$ mm.
\item File size limitation: 2.0 MB.
\item Paper length: %Minimum 4 pages and m
Maximum 8 pages, including figures, tables and references.
In exceptional circumstances up to two additional pages will be
permitted for a charge of \$150 per additional page.
\item Paper formatting: Double column, single spaced, 10pt font.
\item Text width: 7.0 in (178 mm) and text height: 9.375 in (240 mm).

All text and figures must be contained in the $178 \times 240$ mm  image area.
\item The left/right/bottom margin must be 0.75 in (19 mm).
\item The top margin must be 0.75 in (19 mm),
 except for the title page where it must be 1 in (25 mm).
\item Text should appear in two columns, each 3.4 in (86.5 mm) wide with 0.2 in
(5 mm) space between columns.
%\item On the first page, the top 50 mm (2 in) of both columns is reserved for 
%the title, author(s), and affiliation(s) as shown in this sample paper. 
%These items should be centered across both columns, starting at 35 mm 
%(1.375 inches) from the top of the page.
\item Do NOT page number your manuscript.
\item Unix LaTeX users please use the following command:
\begin{itemize}
\item latex mypaper
\item dvips -Ppdf -G0 -tletter mypaper.dvi
\item ps2pdf mypaper.ps mypaper.pdf
\end{itemize}
\end{itemize}
    
The page size and margin settings in IEEEtran.cls are set for 
IEEE Transactions papers. We have made some adjustments to 
produce this sample paper.

{\it Also, please note that IEEE PDF eXpress will be made available to assist you in 
creating the IEEE Xplore compliant PDF file for the camera-ready submissions.}

% An example of a double column floating figure using two subfigures.
%(The subfigure.sty package must be loaded for this to work.)
% The subfigure \label commands are set within each subfigure command, the
% \label for the overall fgure must come after \caption.
% \hfil must be used as a separator to get equal spacing
%
%\begin{figure*}
%\centerline{\subfigure[Case I]{\includegraphics[width=2.5in]{subfigcase1}
% where an .eps filename suffix will be assumed under latex, 
% and a .pdf suffix will be assumed for pdflatex
%\label{fig_first_case}}
%\hfil
%\subfigure[Case II]{\includegraphics[width=2.5in]{subfigcase2}
% where an .eps filename suffix will be assumed under latex, 
% and a .pdf suffix will be assumed for pdflatex
%\label{fig_second_case}}}
%\caption{Simulation results}
%\label{fig_sim}
%\end{figure*}

\section{Main Results}
The main results and findings go here. You may also have a section for 
Preliminaries before this section.

First, if you do not want to number an equation, do not use
$\backslash$begin--$\backslash$end.
You can either use $\backslash$[ --$\backslash$] or
\$\$--\$\$. For example, we have
$$\dot x= f( x,u) + g (x,u)$$
or 
\[\ddot s=G(s,t)\]
where $f,$ $g,$ and $G$ are functions. It is recommended that you do not 
number an equation if it will not be cited in your paper.

Equation (\ref{eq:eq1}) is numbered!
The following equation is produced 
using $\backslash$begin\{equation\}--$\backslash$end\{equation\}.
The main objective function for each unit can be represented by a 
quadratic cost function given by
\begin{equation}
F_i(P_i)=a_{i}+b_{i}P_i+c_{i}P_{i}^2
\label{eq:eq1}
\end{equation}
where  $a_{i},\ b_{i}$, and $c_{i}$ in (\ref{eq:eq1}) are the fuel consumption cost coefficients 
of unit $i$, and $P_i$ represents the value of the power to be 
determined for unit $i$.

Recently, it is popular to use
$\backslash$begin\{align\}--$\backslash$end\{align\}
instead of
$\backslash$begin\{eqnarray\}--$\backslash$end\{eqnarray\}.
Equation (\ref{eq4b}) is produced using 
$\backslash$begin\{align\}--$\backslash$end\{align\}.
The objective function for each unit can be represented by
\begin{align}
\dot {x}_l=& \sum_{i = 1}^m {\frac{c_{P_{x_i} } e^{k_{x_i}\bar{x}_i} + c_{N_{x_i} }
e^{ -  k_{x_i} \bar{x}_i}}{e^{k_{x_i} \bar{x}_i} + e^{ - k_{x_i} \bar{x}_i}}} \nonumber\\
& + \frac{1}{2}\sum\limits_j^q (c_{P{u_j }} + c_{N _{u_j }} ) \nonumber\\
y=& \ A_0 + A_1 \tanh (K_x \bar {x}) + B\tanh (K_u \bar {u}) \nonumber\\
 =& \ F(x),\label{eq4b}
\end{align}
where $F(x)$ is a function.

Well, the same equation, when it is produced using 
$\backslash$begin\{eqnarray\}--$\backslash$end\{eqnarray\} becomes
(\ref{eq4c}):
\begin{eqnarray}
\dot {x}_l&=& \sum_{i = 1}^m {\frac{c_{P_{x_i} } e^{k_{x_i}\bar{x}_i} + c_{N_{x_i} }
e^{ - k_{x_i} \bar{x}_i}}{e^{k_{x_i} \bar{x}_i} + e^{ - k_{x_i} \bar{x}_i}}} \nonumber\\
&&+ \frac{1}{2}\sum\limits_j^q (c_{P{u_j }} + c_{N _{u_j }} ) \nonumber\\
y&=& \ A_0 + A_1 \tanh (K_x \bar {x}) + B\tanh (K_u \bar {u})\nonumber\\
&=& \ F(x),\label{eq4c}
\end{eqnarray}
where $F(x)$ is a function. You get the idea!

\subsection{Example of a Figure}
An example of a floating figure using the graphicx package.
Note that $\backslash$label must occur AFTER (or within) $\backslash$caption.
For figures, $\backslash$caption should occur after the 
$\backslash$includegraphics.
You also need to know how to cite your figure. Here is an example:
Figure~\ref{fig_sim} show our simulation results.

\begin{figure}[htp]
\centerline{\includegraphics[width=3.35in]{\Pic{eps}{fig1}}}
\caption{Simulation results}
\label{fig_sim}
\end{figure}

\subsection{Figures and Tables}
Please follow the style in the sample paper when generating your figures and tables.

\def\del{
\subsection{What Sections to Include}
Usually, your paper should have Introduction, Main Results, Simulation Results, and
Conclusions. You may also add Acknowledgments if you like.
After that, you should have your References.}

\subsection{Page Limit and Overlength Page Charges}

A paper submitted to this conference should be prepared in a single-spaced, two-column 
format and its length must be kept to 8 pages and below.
In exceptional circumstances up to two additional pages will be
permitted for a charge of \$150 per additional page.
Table~\ref{table_example} shows the page limit
and page charge schedule.

% An example of a floating table. Note that, for IEEE style tables, the 
% \caption command should come BEFORE the table.Table text will default to
% \footnotesize as IEEE normally uses this smaller font for tables.
% The \label must come after \caption as always.
%
\begin{table}
\begin{center}
%% increase table row spacing, adjust to taste
\renewcommand{\arraystretch}{1.3}
\caption{Page Limit}
\label{table_example}
% The array package and the MDW tools package offers better commands
%% for making tables than plain LaTeX2e's tabular which is used here.
\begin{tabular}{|c|c|}
\hline
Page limits & 8\\
\hline
Excess page charge & \$150/page\\
\hline
\end{tabular}
\end{center}
\end{table}

Another example of table is shown in Table~\ref{tab-liu2}.

\begin{table}[h]
\caption{Comparison results  with methods in \cite{cit:mic77} 
(40 unit system with valve-point effects)}
\begin{center}
\begin{tabular}{|c|c|c|c|c|c|}
\hline
\multicolumn{1}{|c|}{\raisebox{-1.50ex}[0cm][0cm]{\!Method\!}}
& \multicolumn{1}{|c|}{Mean}% time}
& \multicolumn{1}{|c|}{Best}% time}
& \multicolumn{1}{|c|}{Mean}% cost}
& \multicolumn{1}{|c|}{Maximum}% cost}
& \multicolumn{1}{|c|}{Minimum} \\% cost}\\
%\hline
& time&time& cost&cost&cost\\ \hline
 CEP   &  $928.36$  &  $926.20$  &  $124793.5$ & $126902.9$ & $123488.3$ \\ \hline
 FEP   &  $646.16$  &  $644.28$  &  $124119.4$ & $127245.9$ & $122679.7$ \\ \hline
 MFEP  &  $1056.8$  &  $1054.2$  &  $123489.7$ & $124356.5$ & $122647.6$ \\ \hline
 IFEP  &  $632.67$  &  $630.36$  &  $123382.0$ & $125740.6$ & $122624.4$ \\ \hline
 TM    &  $94.28$  &  $91.16$  &  $123078.2$ & $124693.8$ & $122477.8$ \\ \hline
\end{tabular}
\label{tab-liu2}
\end{center}
\end{table}
